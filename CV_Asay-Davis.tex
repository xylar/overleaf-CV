\documentclass[12pt,letterpaper]{article}
\usepackage[T1]{fontenc}
%\usepackage[pagestyles]{titlesec}
\usepackage{txfonts}
%\usepackage{blindtext}
\usepackage{color}
\usepackage{fancyhdr}
\usepackage[top=1in, bottom=1in, left=1in, right=1in] {geometry}
\usepackage{mdwlist}
\usepackage{tabularx}
\usepackage{enumitem}
\usepackage{multirow}

\definecolor{titleBlue}{RGB}{40,66,157}



\pagestyle{fancy}

\raggedright

\begin{document}

\setlength{\parindent}{0in}

\pagestyle{fancy}
\renewcommand{\headrulewidth}{0.0pt}


\begin{center}
{\bf \color{titleBlue} DR. XYLAR S. ASAY-DAVIS}\\
\end{center}

\begin{tabularx}{\textwidth}{@{}  X r @{} }
  Los Alamos National Laboratory &  +1 505-667-0603 \\
  Solid Mechanics and Fluid Dynamics &  xylar@lanl.gov\\
  P.O. Box 1663, T-3, MS-B216\\
  Los Alamos, NM, 87545, USA \\
\end{tabularx}

\begin{flushleft}
\subsubsection*{\color{titleBlue} EDUCATION}

\begin{tabularx}{\textwidth}{@{}  l X r @{} }
PhD & {\bf University of California, Berkeley} & Berkeley, CA, US\\
& Applied Science and Technology & Dec. 2008 \\
\\
BS & {\bf University of California, San Diego} & La Jolla, CA, US \\
& {\it summa cum laude} Physics and Electrical Engineering & May 2001\\
\end{tabularx}

\subsubsection*{\color{titleBlue} RESEARCH EXPERIENCE}

\begin{tabularx}{\textwidth}{@{} X r @{}}
{\bf Los Alamos National Laboratory} &  Los Alamos, NM, USA \\
Staff scientist  &   Sep. 2017--present \\
\end{tabularx}

\vspace{4pt}

\textbf{E3SM} (Energy Exascale Earth System Model) and \textbf{MPAS} (Model for Prediction Across Scales): primary developer of ice-shelf cavities and land ice--ocean coupling, and of MPAS-Analysis.

\begin{itemize}[noitemsep,nolistsep]
  \item Implemented land ice-ocean boundary conditions and boundary current parameterization
  \item Developed new vertical coordinate compatible with terrain-following coordinate 
  \item Designed and built idealized test cases with ice-shelf cavities for model evaluation
  \item Developing a simulation analysis system (MPAS-Analysis) for ocean, land ice and sea ice components of E3SM
\end{itemize}
\vspace{4pt}

\textbf{MISOMIP} (Marine Ice Sheet-Ocean Model Intercomparison Project): co-chair

\begin{itemize}[noitemsep,nolistsep]
  \item Designed and performed ice sheet--ocean experiments (Asay-Davis et al. 2016)
  \item Coordinating analysis of MIP results
\end{itemize}

\vspace{4pt}

\textbf{POPSICLES} coupled ice sheet-ocean model: primary developer

\begin{itemize}[noitemsep,nolistsep]
  \item Couples the BISICLES ice sheet model and the Parallel Ocean Program (POP) v. 2x
  \item Both idealized and pan-Antarctic/Southern Ocean simulations
  \item Exploring the effects of atmospheric forcing on melting, ice dynamics and mass loss
\end{itemize}

\vspace{10pt}

\begin{tabularx}{\textwidth}{@{} X r @{}}
{\bf Potsdam Institute for Climate Impact Research} &  Potsdam, Germany\\
Research scientist  &   Sep. 2014--Sep. 2017 \\
Guest scientist &   Nov. 2012--Aug. 2014, Sep. 2107--present
\end{tabularx}

\vspace{4pt}

\begin{tabularx}{\textwidth}{@{} X r@{} }
{\bf  New York University}  & New York, NY, US\\
Postdoctoral Associate. Advisor: David M. Holland &  Feb.--Dec. 2013
\end{tabularx}

\vspace{4pt}

Regional studies of ice sheet-Ocean interactions around Antarctica
\vspace{10pt}

\begin{tabularx}{\textwidth}{@{} X r@{} }
{\bf Los Alamos National Laboratory} & Los Alamos, NM, US \\
Postdoctoral Researcher. Climate, Ocean and Sea Ice Modeling Group & Oct. 2009--Jul. 2012
\end{tabularx}

\vspace{4pt}

Development and testing of ice shelf--ocean interactions in POP2x
\vspace{10pt}

\begin{tabularx}{\textwidth}{@{} X r@{} }
{\bf University of California, Berkeley} & Berkeley, CA, US \\
Postdoctoral Researcher. Advisors: Philip S. Marcus and Imke de Pater &  Aug. 2008--Aug. 2009 
\end{tabularx}

\vspace{4pt}

Study of Jupiter's atmosphere using remote sensing methods

\subsubsection*{\color{titleBlue} DISSERTATION}

{\bf Vortex Flows in Jupiter's Atmosphere and in Protoplanetary Disks}

Philip S. Marcus (chair), Tarek I. Zohdi, Eugene Chiang

\vspace{5pt}

{\bf Part 1: Extracting vortex velocities in Jupiter's atmosphere }

I developed a new method for producing accurate velocity fields Jupiter's weather layer.  My colleagues and I use these velocity fields in numerical models to study Jupiter's climate.

\vspace{5pt}

{\bf Part 2: Trapping dust in 3D vortices in a the disk around a new star}

Using 3D numerical simulations, I studied the formation and evolution of vortices in the disk around a newly formed star.  I showed that these vortices are feasible birthplaces for planets.

\subsubsection*{\color{titleBlue} PROGRAMMING EXPERIENCE}

\begin{itemize}[noitemsep,nolistsep]
  \item Programming languages: C++, Fortran, Python
  \item Parallel libraries: MPI, parallel I/O
  \item Visualization: Paraview, VTK, Matplotlib, Matlab
  \item Open science and collaboration tools: git, GitHub, GitLab, Open Science Framework, Confluence, Overleaf
\end{itemize}

\subsubsection*{\color{titleBlue} FELLOWSHIP}

{\bf National Science Foundation Fellowship}, 2002-2005.

\subsubsection*{\color{titleBlue} PROFESSIONAL TRAININGS}

{\bf Summer Institute for Preparing Future Faculty}, 
University of California, Berkeley, 2008.

{\bf Advanced Climate Dynamics Courses},
Lyngen, Norway, Summer 2010.

\subsubsection*{\color{titleBlue} COLLABORATORS} 

\begin{tabular}{@{} l l l@{} }
\rule{-4pt}{16pt} Prof. Ricarda Winkelmann/PISM Team & PIK &  Potsdam, Germany \\
\rule{-4pt}{16pt} Dr. Daniel Martin & LBNL & Berkeley, CA, US\\
\rule{-4pt}{16pt} Prof. David Holland & NYU & New York, NY, US \\ 
\rule{-4pt}{16pt} Dr. Stephen Cornford & U. Bristol & Bristol, UK\\
\rule{-4pt}{16pt} ISSM/MITgcm Team & JPL & Pasadena, CA, US\\
\rule{-4pt}{16pt} MOM6 Ice Sheet-Ocean Team & GFDL & Princeton, NJ, US\\
\rule{-4pt}{16pt} MITgcm Ice Sheet-Ocean Team & BAS/U. Edinburgh & Cambridge/Edinburgh, UK\\
\rule{-4pt}{16pt} Elmer Ice/NEMO Team & IGE/U. Grenoble Alpes & Grenoble, France\\
\rule{-4pt}{16pt} Elmer Ice/ROMS Team & U. Tasmania & Hobart, Tas., Aus.\\
\rule{-4pt}{16pt} BISICLES/NEMO Team & U. Reading/UK MetOffice & Reading/Exciter, UK\\
\rule{-4pt}{16pt} FVCOM Team & Akvaplan-niva & Troms{\o}, Norway\\
\end{tabular}

\subsubsection*{\color{titleBlue} MENTORING} 

\begin{tabularx}{\textwidth}{@{} X r@{} }
{\bf Gunter Leguy} &  Jun. 2010--Jul. 2015 \\
PhD Student, New Mexico Tech \& Los Alamos National Laboratory & Los Alamos, NM, US
\end{tabularx}

\vspace{4pt}

Gunter's work under joint supervision by myself, Prof. William Stone (NMT) and Dr. William Lipscomb (LANL) focused on studying numerical parameterizations of basal friction near grounding lines in ice sheet models. 

\subsubsection*{\color{titleBlue} PHD COMMITTEE} 

\begin{tabularx}{\textwidth}{@{} X r@{} }
{\bf Ignacio Merino} &  Dec. 2016 \\
IGE \& Univ. Grenoble Alpes & Grenoble, France
\end{tabularx}
\vspace{4pt}

I acted as an external reviewer and examiner.


\subsubsection*{\color{titleBlue} INVITED PRESENTATIONS}
\vspace{-5pt}
\begin{enumerate*}
\item X.S. Asay-Davis, S.L. Cornford, D.F. Martin, G.H. Gudmundsson, D.M. Holland,
and D. Holland. Design of the MISMIP+, ISOMIP+, and MISOMIP ice-sheet, ocean, and coupled ice sheet-ocean intercomparison projects. Invited presentation. European Geosciences Union General Assembly, Vienna, Austria, April 12-17, 2015.
\item X. Asay-Davis, D. Martin, S. Cornford, S. Price, and W. Collins. Coupled ice sheet-ocean simulations with the POPSICLES model. Invited presentation. General Assembly of the International Union of Geodesy and Geophysics, Prague, Czech Republic, June 22-July 2, 2015.
\item X. Asay-Davis, B. K. Galton-Fenzi, D. E. Gwyther, N. C. Jourdain, D. F. Martin, Y. Nakayama, H. Seroussi. Results from ISOMIP+ and MISOMIP1, two interrelated marine ice sheet and ocean model intercomparison projects. Invited presentation. Fall Meeting of the American Geophysical Union, San Francisco, CA, US, Dec. 12-16, 2016.
\end{enumerate*}

\subsubsection*{\color{titleBlue} PUBLICATIONS}
\vspace{-5pt}

\begin{enumerate*}
\item Petersen, M., \textbf{X. Asay-Davis}, A. Berres, D. Comeau, N. Feige, D. Jacobsen, P. Jones, M. Maltrud, T. Ringler, G. Streletz, A. Turner, L. Van Roekel, M. Veneziani, J. Wolfe, P. Wolfram and J. Woodring. (2018). An evaluation of the ocean and sea ice climate of E3SM using MPAS and interannual CORE-II forcing.  \textit{JAMES}, in prep.
\item R. Reese, T. Albrecht, M. Mengel, \textbf{X. Asay-Davis}, and R. Winkelmann (2018): Antarctic sub-shelf melt rates via PICO, \textit{The Cryosphere}, accepted.
\item \textbf{X. S. Asay-Davis}, N. C. Jourdain, Y. Nakayama (2017). Developments in Simulating and Parameterizing Interactions Between the Southern Ocean and the Antarctic Ice Sheet. \textit{Current Climate Change Reports} 3(4), 316-329. 
\item M. S. Dinniman, \textbf{X. S. Asay-Davis}, B. K. Galton-Fenzi, P. R. Holland, A. Jenkins, and R. Timmermann (2016). Modeling Ice Shelf/Ocean Interaction in Antarctica. \textit{Oceanography}, 29(4), 144-153. 
\item \textbf{X. S. Asay-Davis}, S.L. Cornford, G. Durand, B.K. Galton-Fenzi, R.M. Gladstone, G.H. Gudmundsson, T. Hattermann, D.M. Holland, D. Holland, P.R. Holland, D.F. Martin, D. P. Mathiot, F. Pattyn, and H. Seroussi (2016). Experimental design for three interrelated marine ice sheet and ocean model intercomparison projects: MISMIP v. 3 (MISMIP+), ISOMIP v. 2 (ISOMIP+) and MISOMIP v. 1 (MISOMIP1). \textit{Geoscientific Model Development}, 9(7), 2471-2497.
\item G.R. Leguy, \textbf{X.S. Asay-Davis}, W.H. Lipscomb (2014). Parameterization of basal friction near grounding lines in a one-dimensional ice sheet model. \textit{The Cryosphere}, 8(4), 1239-1259.
\item P.S. Marcus, \textbf{X. Asay-Davis}, M.H. Wong, I. de Pater (2012). Jupiter's New Red Oval: Dynamics, Color, and Relationship to Jovian Climate Change. \textit{Journal of Heat Transfer}, 135(1), 011007.
\item M.H. Wong, P.S. Marcus, I. de Pater, \textbf{X. Asay-Davis}, C.Y. Go (2011). Vertical structure of Jupiter's Oval BA before and after it reddened: What changed? \textit{Icarus}, 215(1), Pages 211-225.
\item \textbf{X.S. Asay-Davis}, P.S. Marcus, M.H. Wong, I. de Pater (2011). Changes in Jupiter's zonal velocity between 1979 and 2008, \textit{Icarus}. 211(2), 1215-1232.
\item A.T. Lee, E. Chiang, \textbf{X. Asay-Davis}, J. Barranco (2010). Forming Planetesimals by Gravitational Instability. I. The Role of the Richardson Number in Triggering the Kelvin-Helmholtz Instability, \textit{The Astrophysical Journal}, 718(2), 1367.
\item A.T. Lee, E. Chiang, \textbf{X. Asay-Davis}, J. Barranco (2010). Forming Planetesimals by Gravitational Instability: II. How Dust Settles to its Marginally Stable State, \textit{The Astrophysical Journal}, 725(2), 1938-1954. 
\item \textbf{X.S. Asay-Davis}, P.S. Marcus, M.H. Wong, I. de Pater (2009). Jupiter's shrinking Great Red Spot and steady Oval BA: Velocity measurements with the ``Advection Corrected Correlation Image Velocimetry'' automated cloud-tracking method, \textit{Icarus}. 203(1), 164-188.
\item S. Shetty, \textbf{X.S. Asay-Davis} and P.S. Marcus (2007). On the interaction of Jupiter's Great Red Spot and zonal jet streams. \textit{Journal of Atmospheric Sciences}. 64, 4432-4444. 
\end{enumerate*} 

\end{flushleft}

\end{document}
